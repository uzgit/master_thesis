Some Gazebo models and worlds have been developed to allow ArduPilot to interact with Gazebo. They are included in \cite{ardupilot_gazebo_plugin}. For example, the Iris quadcopter (used in this project) is a small quadcopter fitted with a gimbal. The gimbal is a simple, 2-dimensional gimbal with two ``revolute'' joints - a yaw joint referred to as \texttt{iris\_gimbal\_mount}, and a pitch joint referred to as \texttt{tilt\_joint}. \cite{ardupilot_gazebo_plugin} specifies an \texttt{iris\_ardupilot} world file which includes the Iris quadcopter and defines a planar world with a runway. A Python script \texttt{sim\_vehicle.py} provides an interface between the Iris and an instance of ArduPilot. The script reads the data from the simulated sensors aboard the Iris and provides throttle signals to its motor plugins. The motor plugins provide simulated thrust in order to realize the effects of the control system and animate the drone model. These allow the drone system to accurately simulate the behavior of the drone as it is controlled by ArduPilot.