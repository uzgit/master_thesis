Robust hardware and software systems exist for enabling autonomous drone flight in the open source community. Several microprocessors have been developed for this purpose, such as the Pixhawk family \cite{pixhawk_website}. Hardware additions, such as the Navio2 \cite{navio2_website}, have been developed as shields for the Raspberry Pi family. This hardware provides critical sensors to flight control software, such as positional data from a \gls{GPS}, air pressure from a barometer (which helps to determine altitude), orientation data from an \gls{IMU}, etc. Power modules regulate battery power to the microprocessor system and also provide useful information pertaining to the system's battery voltage and current draw. Other peripherals can be added through interfaces such as SPI, SBus, analog sensors, UART, 3-pin servo connections, and USB. Telemetry systems enable wireless communication between these vehicles and ground control stations, which provide high-level control such as ``takeoff,'' ``land,'' and point-and-click waypoint selection, as well as system parameter reconfiguration.

In the more primitive environment of a microprocessor such as the Pixhawk, a real time operating system executes the flight control tasks, which can guarantee that critical tasks happen on time. In a fuller operating system such as Raspbian (the Raspberry Pi operating system), the autopilot software typically runs as a system service. Running the autopilot software on a primitive system provides a better guarantee of real time performance of the software, but makes it more difficult to extend the software functionality. Running the autopilot software in a full operating system makes it far easier to add additional functionality by simply running additional programs, but this comes at the price of less real time performance. The autopilot software gathers data from the sensors in order to determine its vehicle's conditions, after which it can determine the proper way to control the vehicle's actuators in order to accomplish its goals. In multirotor drones, the main actuators that the flight control software manages are the motors, which control the drone's attitude. Other typical actuators include gimbals, which are used to aim cameras.

Currently there are 2 main autopilot software distributions which are available open source: ArduPilot and PX4. Different branches of these distributions target different vehicle families, such as quadcopters, hexacopters, octacopters, fixed-wing drones, ground vehicles, and even submarines. They use a lightweight communication protocol called MAVLink, which provides common message sets and allows for efficient data transfer between vehicles and ground control stations. MAVLink itself has several libraries which allow it to interface with other software, such as PyMAVLink (a Python implementation of the message set), MAVproxy (a Python MAVLink server), and MAVROS (which allows MAVLink-enabled vehicles to interface with ROS modules).