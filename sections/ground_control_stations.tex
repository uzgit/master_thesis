

ArduPilot can fully interact with and control the Iris quadcopter in Gazebo. The Iris quadcopter model is, as well as several others, specifically set up to mimic the mechanics of an actual quadcopter, and ArduPilot controls the thrust values applied to the motor mounts of the quadcopter. As in a real life scenario, this is the only means of controlling the attitude, velocity, and acceleration of the drone. An instance of ArduPilot can be launched in order to interact specifically with the Iris within Gazebo using the following command: \texttt{sim\_vehicle.py ArduCopter -f gazebo-iris}. This launches the ``copter'' version of ArduPilot which is further configured using the \texttt{gazebo-iris} configuration file. ArduPilot further opens \gls{UDP} ports in order to communicate with \gls{GCS} instances using the MAVlink communication protocol. In this project, care has been taken to leave both the ArduPilot software and MAVLink protocol unmodified, such that it may be applied to existing drone systems without deep modification.

Several different \gls{GCS} programs are also available open-source. These typically offer a GUI for visualization and temporal tracking of a vehicle's position in a map, forms for adjusting various parameters, displays for sensor values, and, importantly, a concise, GUI-based method of sending MAVLink commands to the vehicle. The QGroundControl software \cite{qgroundcontrol} is the software chosen for interacting with the simulation in this project purely because of its simplicity of installation. Alternatives would also work, but the particular choice of \gls{GCS} software is unimportant.