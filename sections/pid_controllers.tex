\label{section:pid_controllers}
The most common form of feedback process controller is the \gls{PID} controller \cite{pid_design}, and indeed these are the most common feedback controllers used in the ArduPilot software. Given a system with a variable to be controlled, an actuator to provide control effort in order to change the variable, and a sensor to read the state of the controlled variable, a PID controller can produce a smooth and predictable change in the state of the controlled variable, causing it to approach and arrive at a set point. A scalar ($k_p, k_i,$ or $k_d$) controls the influence of each of the controller components (proportional, integral, or derivative, respectively) on the control effort. The effect of the proportional component is to apply a control effort which is proportional to the system's error - that is, the difference between the set point and the current state. This causes the state to approach the set point. The effect of the derivative component is to apply a dampening to the control effort which is proportional to the derivative of the error, in order to slow the state's approach to the set point. The effect of the integral component is to apply a control effort which is proportional to the integral of the error, the benefit of which is to remove persistent, steady state errors which must be identified over time and which are therefore not identifiable by the instantaneous readings of the error or its derivative.

The values of the gains must be tuned for each system to which a PID controller will be applied. Some methods exist to analyze systems in order to determine effective gains \cite{pid_design}, however it is also possible to tune the gains manually with some experience. Aside from the gains, it is also necessary to set some other parameters, such as the minimum control effort, the maximum control effort, and the integral windup limit. The minimum and maximum control effort values provide a means of protecting the actuator or system from excessive control forces which may go beyond physical or process constraints. The integral windup limit sets the maximum effect that the integral component may have on the control effort, motivated by the fact that the integral component can easily cause saturation and unstable system response. PID controller tuning is a solved problem and is therefore not described in great detail in this project. \cite{pid_control} and \cite{pid_design} provide guidelines for this, such as the Ziegler-Nichols method.