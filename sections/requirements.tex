\label{section:requirements}
The landing controller must be robust to the positional noise apparent in normal GPS modules, which can cause mishaps during landing, meaning it must have a positional accuracy on the order of about 0.2 meters when targeting a landing platform with a diameter of 1 meter. It must also allow for landing on both stationary and moving landing platforms. To this end, the landing controller does not use GPS as its main navigational source, but instead uses fiducial markers as mentioned in Section \ref{subsection:fiducial_markers}, to estimate its position with respect to the landing platform. In order to land safely, the landing controller must have a way to identify and stop a failed landing, which could be caused by erratic landing platform movement, drift due to wind, obfuscation of the landing platform's fiducial markers, etc. In order to accomplish this, a descent region is generated, as shown in Figure \ref{fig:descent_region}. The radius of this region increases exponentially in the altitude of the drone above the landing pad. Outside of this region, the drone is not allowed to descend. This gives the drone time to correct its horizontal position and maintain a safe descent. This is outlined in Section \ref{subsection:control_policy}.

The landing platform must be recognizable under a wide variety of conditions. A gimbal-mounted camera increases the range in which the landing platform can be identified by the camera. This requires the addition of a gimbal controller which aims the camera for the duration of the landing, explained in Section \ref{section:gimbal_controller}. A large WhyCon marker allows for easy recognition of the landing platform from long distances, while a smaller April Tag marker provides continuous pose estimation throughout the final descent, when the WhyCon marker is not entirely visible in the camera's field of view.

The landing controller and gimbal controller must be easy to integrate into existing drone systems. As mentioned in Section \ref{subsection:ardupilot}, the chosen autopilot software is ArduPilot, which uses the MAVLink communication protocol. To allow the proposed landing system to be easily integrated into existing drone systems, the ArduPilot code and MAVLink dialects are not edited. The landing controller and gimbal controller are developed as ROS modules which can be run on a companion board, interfacing with ArduPilot for control. On more advanced flight controllers, such as Navio2-enabled Raspberry Pis (which have full operating systems), these ROS modules can be run on the same board as the ArduPilot software, as system services. In order to abstract from primitive motor throttle commands and specific drone body types, the landing controller controls the drone using only high-level velocity set points. This leaves the low-level motor control to ArduPilot itself. The hardware tool set is also kept simple (just a gimbal-mounted camera and a companion board) for this requirement.