ROS (Robot Operating System) \cite{ros} is an open source framework for developing modular, flexible, robust robotic software. A large variety of modules, libraries, and conventions are included in the \gls{ROS} umbrella. Functional parts of robotic applications are divided into \gls{ROS} modules and libraries, whose definitions provide lists of dependencies and installation instructions, as well as application code. The \texttt{roscore} node (process) provides a means for additional nodes to communicate using \textit{topics}. Topics can be of any data type - such as Boolean variables, integer variables, float variables, images, custom types, etc. Nodes are launched either independently or using \texttt{.launch} files, which specify pre-launch requirements (such as prerequisite nodes), launch parameters, and which may launch additional nodes. A key benefit to using \gls{ROS} is that there are many existing packages supplying many different functionalities, such as \gls{PID} controllers (see Section \ref{section:pid_controllers}), fiducial marker frameworks (see Section \ref{subsection:fiducial_markers}), drivers for cameras (which provide the camera's image as a topic), and many useful data structures such as quaternions (see Section \ref{subsection:quaternions}), vectors. Another important library is the Transform 2 library (TF2) \cite{tf2}, which provides invaluable support for coordinate system transforms. Using this library, nodes can generate time-stamped representations of the positions and orientations of relevant physical components of the robot. These transforms are tagged by name and organized into a tree, after which point, they can be composed with one another to derive further transformations. This is useful in determining the positions of components on the robot itself, and also objects in space.